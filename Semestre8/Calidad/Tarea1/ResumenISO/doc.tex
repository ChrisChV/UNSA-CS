\documentclass[a4paper,12pt]{article}
\usepackage[utf8]{inputenc}
\usepackage[spanish]{babel}
\usepackage{color}
\usepackage{parskip}
\usepackage{graphicx}
\usepackage{multirow}
\usepackage{listings}
\usepackage{vmargin}
\usepackage{datetime}
\newdate{date}{5}{09}{2017}
\graphicspath{ {imagenes/} }
\definecolor{mygreen}{rgb}{0,0.6,0}
\definecolor{lbcolor}{rgb}{0.9,0.9,0.9}
\usepackage{epstopdf}


\setpapersize{A4}
\setmargins{2.5cm}       % margen izquierdo
{1.5cm}                        % margen superior
{16.5cm}                      % anchura del texto
{23.42cm}                    % altura del texto
{10pt}                           % altura de los encabezados
{1cm}                           % espacio entre el texto y los encabezados
{0pt}                             % altura del pie de página
{2cm}     



\begin{document}
\title{Resumen NTP-ISO/IEC 29110}
\author{
Christofer Fabián Chávez Carazas \\
\small{Universidad Nacional de San Agustín de Arequipa} \\
\small{Escuela Profesional de Ciencia de la Computación} \\
\small{Calidad de Software}
}
\date{\displaydate{date}}

\maketitle

\begin{enumerate}
 \item \textbf{Alcance}
 
 La NTP-ISO/IEC 29110 es aplicable a las Pequeñas Organizaciones (PO). Las PO son empresas, organizaciones, departamentos o proyectos de hasta 25 personas.
 La ISO no pretende excluir a organizaciones grandes, ya que estas se pueden dividir en muchas POs. La ISO proporciona una Guía de Gestión e Ingeniería para el Perfil Básico.
 El Perfil Básico describe el desarrollo de software de una sola aplicación por un sólo equipo de proyecto sin ningún riesgo especial o factores situacionales.
 Esta parte está diseñada para ser utilizada con cualquiera de los procesos, técnicas y métodos que mejoren la satisfacción y la productividad de los clientes de la PO.
 
 \item \textbf{Visión general}
 
 La ISO está destinada para ser usada por la PO para establecer procesos para implementar cualquier enfoque o metodología de desarrollo, incluyendo, por ejemplo, ágil, evolutivo,
 incremental, desarrollo dirigido por pruebas. La PO puede obtener los siguientes beneficios:
 \begin{itemize}
  \item Un conjunto acordado de requisitos del proyecto y productos esperados.
  \item Un proceso de gestión disciplinado.
  \item Un proceso sistemático de implementación de Software que satisfaga las necesidades del Cliente.
 \end{itemize}
 Para el uso de la guía se necesita que el enunciado de Trabajo del proyecto esté documentado, la viabilidad del proyecto fue realizada antes de su inicio y el equipo del proyecto está asignado
 y entrenado.
  
 \item \textbf{Proceso de Gestión del proyecto}
 
 El propósito del proceso Gestión del Proyecto es establecer y llevar a cabo de manera sistemática las Tareas de un proyecto de implementación de Software, que permitan cumplir
 con los Objetivos del proyecto en calidad, tiempo y costos esperados. Los objetivos para el proceso Gestión del Proyecto son los siguientes:
 \begin{itemize}
  \item \textbf{GP.O1} El Plan del Proyecto para el plan del proyecto es desarrollado de acuerdo al Enunciado de Trabajo y revisado y aceptado por el Cliente. Las Tareas y los Recursos
  necesarios para completar el trabajo son dimensionados y estimados.
  \item \textbf{GP.O2} El avance del proyecto es monitoreado contra el Plan del Proyecto y registrados en el Registro de Estado del Avance.
  \item \textbf{GP.O3} Los cambios a los requisitos de Software son evaluados por su impacto técnico, en costo y en el cronograma.
  \item \textbf{GP.O4} Reuniones de revisión con el Equipo de trabajo y el Cliente son realizadas. Los acuerdos que surgen de estas reuniones son documentadas y se les hace seguimiento.
  \item \textbf{GP.O5} Los riesgos son identificados en el desarrollo y durante la realización del proyecto.
  \item \textbf{GP.O6} Una estrategia de Control de Versiones de Software es desarrollada.
  \item \textbf{GP.O7} El Aseguramiento de Calidad del Software es realizado.
 \end{itemize}
 El Proceso de Gestión del Proyecto consiste en las siguientes actividades:
 \begin{itemize}
  \item GP.1 Planificación del Proyecto
  \item GP.2 Ejecución del Plan del Proyecto
  \item GP.3 Evaluación y Control del Proyecto
  \item GP.4 Cierre del Proyecto
 \end{itemize}
 La actividad de Planificación del Proyecto documente los detalles de la planificación necesarios para gestionar el proyecto.
 La actividad de la ejecución del Plan del Proyecto implementan el plan documentado en el proyecto.
 La actividad de Evaluación y Control del Proyecto evalúa el desempeño del plan contra los compromisos documentados.
 La actividad de Cierre del Proyecto proporciona documentación y productos del proyecto de acuerdo con los requisitos del contrato.


 
  
\end{enumerate}


\end{document}

