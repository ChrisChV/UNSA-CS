\documentclass[a4paper,12pt]{article}
\usepackage[utf8]{inputenc}
\usepackage[spanish]{babel}
\usepackage{color}
\usepackage{parskip}
\usepackage{graphicx}
\usepackage{multirow}
\usepackage{listings}
\usepackage{vmargin}
\graphicspath{ {imagenes/} }
\definecolor{mygreen}{rgb}{0,0.6,0}
\definecolor{lbcolor}{rgb}{0.9,0.9,0.9}
\usepackage{epstopdf}


\setpapersize{A4}
\setmargins{2.5cm}       % margen izquierdo
{1.5cm}                        % margen superior
{16.5cm}                      % anchura del texto
{23.42cm}                    % altura del texto
{10pt}                           % altura de los encabezados
{1cm}                           % espacio entre el texto y los encabezados
{0pt}                             % altura del pie de página
{2cm}     

\lstset{
backgroundcolor=\color{lbcolor},
    tabsize=4,    
%   rulecolor=,
    language=[GNU]C++,
        basicstyle=\tiny,
        aboveskip={1.5\baselineskip},
        columns=fixed,
        showstringspaces=false,
        extendedchars=false,
        breaklines=true,
        prebreak = \raisebox{0ex}[0ex][0ex]{\ensuremath{\hookleftarrow}},
        frame=single,
        showtabs=false,
        showspaces=false,
        showstringspaces=false,
        identifierstyle=\ttfamily,
        keywordstyle=\color[rgb]{0,0,1},
        commentstyle=\color[rgb]{0.026,0.112,0.095},
        stringstyle=\color{red},
        numberstyle=\color[rgb]{0.205, 0.142, 0.73},
%        \lstdefinestyle{C++}{language=C++,style=numbers}’.
}

\begin{document}

\title{Divide-and-Conquer}

  Recall that in divide-and-conquer, we solve a problem recursively, applying three steps at
  catch level of the recursion. When the subproblems are large enough to solve recursively,
  we call that the recursive case.
 
  \subsection{Recurrences}
  
  \subsection{Technicalities in recurrences}
  
  \section{The maximum-subarray problem}
  
    \subsection{A brute-force solution}
    
    \subsection{A transformation}
    
    \subsection{A solution using divide-and-conquer}
    
    \subsection{Analyzing the divide-and-conquer algorithm}
    
  \section{Strassen's algorithm for matrix multiplication}
  
    \subsection{A simple divide-and-conquer algorithm}
    
    \subsection{Strassen's method}
    
  \section{The substitution method for solving recurrences}
  
    \subsection{Making a good guess}
    
    \subsection{subtleties}
    
    \subsection{Avoiding pintfalls}
    
    \subsection{Changing variables}
    
  \section{The recursion-tree method for solving recurrences}
    


\end{document}
