\documentclass[a4paper,12pt]{article}
\usepackage[utf8]{inputenc}
\usepackage[spanish]{babel}
\usepackage{color}
\usepackage{parskip}
\usepackage{graphicx}
\usepackage{multirow}
\usepackage{listings}
\usepackage{vmargin}
\graphicspath{ {imagenes/} }
\definecolor{mygreen}{rgb}{0,0.6,0}
\definecolor{lbcolor}{rgb}{0.9,0.9,0.9}
\usepackage{epstopdf}


\begin{document}
\title{Resúmenes de Abstracts e Introducciones}
\author{
Christofer Fabián Chávez Carazas \and Ruben Torres Lima \\
\small{Universidad Nacional de San Agustín} \\
\small{Proyectos I}
}

\maketitle

\section{Resúmenes}

\subsection{(Paper 1) An Efficient Parallel Algorithm for Segured Data Communications Using RSA Public Key Cryptography Method}

RSA en uno de los algoritmos basados en PKC(public-key pryptography) más importantes. Está basado en una técnica de
factorización que da como resultado números muy grandes. Manejar esos números directamente en una infraestructura GCC es imposible.
El paper propone un algoritmo paralelo para RSA, con el objetivo de poder manejar los números y las operaciones en
la infraestructura GCC.

\subsection{(Paper 2) A Novel Image Encryption Algorithm using AES and Visual Cryptography}

El principal objetivo de la encriptación de imágenes es transmitir una imagen con seguridad por una red. Ya existen
algoritmos que hacen esto, pero no aseguran la \textit{key} o no son muy amigables con el hardware 
actual. El paper propone un algoritmo que asegura la key usada en AES convirtiéndola en una imagen y dividiéndola en
$n$ \textit{shares} usando técnicas de \textit{Visual Secret Sharing}.

\subsection{(Paper 3) A New Technique for Color Share Generation using Visual Cryptography}

La criptografía visual encripta la imagen en \textit{shares} y la desencripta apilando todos los \textit{shares} para revelar
la imagen. En el paper proponen un esquema para generar \textit{shares} con el color de la imagen utilizando los
componentes R,G y B.

\section{Análisis y Comparación de los Abstracts}

\subsection{Presentación del Problema}

Los autores presentan el problema de maneras diferentes, algunos no lo hacen tan extenso, por ejemplo en el Paper 3 se presenta así: \par
\textit{Now a days, most of our data is travelled over
internet so the security of that data is most important. Visual
cryptography plays a very crucial role for security of image
based secret.} \par
y en el Paper 2 sólo son unas cuantas líneas: \par
\textit{With the current emergence of the Internet,
there is a need to securely transfer images between
systems.} \par
Tal vez no se explayan en el problema aquí porque lo hacen en la introducción. En cambio, en el Paper 1 casi la mitad del 
abstract trata de exponer el problema: \par
\textit{Public-key infraestructure based cryptographic algorithms are
usually considered as slower than their corresponding symetric key
based algorithms due to their root in modular arithmetic [...] the sequential
implementation of RSA becomes compute-intensive and takes a lot of time and energy to execute.
Moreover, it is very difficult to perform intense modular computations on very
large integers because of the limitation in size of basic data types available with
GCC infraestructure.} \par

Estos tres papers comienzan con esto. Nos cuentan cuál es el problema y las necesidades del área en done están trabajando
para luego darnos una propuesta.

\subsection{Propuesta}

En esta parte los autores te presentan lo que han hecho. Alguno autores se echan flores y se explayan en su presentación, como
en el Paper 1: \par
\textit{In this paper, we are looking into the possibility of improving
the performance of proposed parallel RSA algorithm by using two different techniques,
first implementing modular calculations on larger integers using GMP library and
second by parallelizing it using OpenMP on the GCC infraestructure.} \par
otros son más moderados, como en el Paper 2: \par
\textit{In this context, we propose a secure image
encryption algorithm that uses both AES and Visual
Cryptographic techniques to protect the image.} \par
y otros sólo utilizan una línea para presentarlo, como en el Paper 3: \par
\textit{Here we are proposing a new scheme for
color share generation.} \par
pero luego explican un poco su propuesta para hacerla más interesante, como en el Paper 3: \par
\textit{In this scheme R, G and B component is
extracted from color image then apply gray share generation
algorithm on R component and make n number of R gray shares
then all shares are combine with B and G component to make
color shares.} \par
o como en el Paper 2: \par
\textit{The image
is encrypted using AES and an encoding schema has been
proposed to convert the key into shares based on Visual
Secret Sharing.} \par
Al ser esto parte del Abstract los autores no tienen mucho espacio para explayarse, eso sí se hace en la introducción, por eso
las presentaciones de algunas propuestas no son muy amplias, pero aún así tienen que verse interesantes para poder enganchar
al lector.

\section{Análisis y comparación de las Introducciones}


\subsection{Contextualización}

Todos los papers empiezan contextualizando al lector en el campo de investigación del autor. Introducen diciendo por qué es
importante su tema, como en el Paper 3: \par

\textit{In today’s world data security is very important because
most of our data is travelled over internet. [...]} \par

dan algunas aplicaciones, como en el Paper 2: \par

\textit{Image
encryption has applications in many fields including Banking,
Telecommunication and Medical Image Processing etc. [...]} \par

y citan algunos trabajos en el área, como en el Paper 1 y el Papper 3: \par

\textit{One of the most important techniques is public-key cryptography (PKC)
or asymetric cryptography which was invented by Whirfield Diffie, Martin Hellman [2]
and Ralph Merkle [3] [...]} \par

\textit{Research shows that using AES and Visual Cryptography to
encrypt images is not entirely new. In [2], an encryption
scheme has been proposed that splits the Image into R, G, and
B components and encrypt them using AES.} \par

No todos los papers tiene esto, por ejemplo, en el Paper 3 en ninguna parte de la introducción se citan trabajos pasados, porque
la siguiente sección esta más centrada a eso. Otro ejemplo es el Paper 1, en donde no se menciona las aplicaciones del tema;
tal vez porque ya son muy bien conocidas las aplicaciones del RSA. Pero sí todos nos cuentan cuál es la importancia de su trabajo.

\subsection{Problema}

Aquí los autores amplían un poco más el problema, no sólo describiéndolo sino también contando los problemas que tienen
trabajos anteriores, como en el Paper 2: \par

\textit{Another interesting approach to encrypt image is
using the chaotic theory based algorithms. [...] However, these are relatively
newer algorithms and in certain systems where hardware
encryption circuits for default algorithms like AES are built in,
these kind of algorithms need an entirely new update of hardware.} \par

En el Paper 1, el autor se toma un párrafo completo para explicar las complicaciones que tiene el RSA secuencial, y el por qué
su trabajo es un aporte importante: \par

\textit{[...] it is imposible to work on such large numbers on GCC infraestructure directly.[...]} \par

En el Paper 3 no se muestra muy claramente cuál es el problema, sólo nos vuelve a recalcar la importancia de su trabajo: \par

\textit{Visual
cryptography plays a very important role for image security.
In this technique image is encrypted in to number of shares
and at decryption side all or some of the shares are overlapped
with each other to reveal the secret image.} \par

\subsection{Propuesta}

En el Paper 3, desde la mitad hasta casi el final da su propuesta y explica qué es lo que hace: \par

\textit{In this paper color image is
taken as an input to the system then we extract the R, G, and B
component from color image. After extraction phase gray
share generation algorithms is applied on only R component
and generate n number of R gray shares. [...]} \par

El Paper 1 da su propuesta al final de la introducción, pero no lo hace de manera directa sino da a conocer la importancia
de las tecnologías que va a usar para resolver el problema: \par

\textit{Recently, the use of OpenMP [6] on the GCC infraestructure for general purpose
computing has been gaining widespread usage for parallelizing algorithms. [...]}

El Paper 2 también da su propuesta al final de la introducción y sí lo hace de manera directa:

\textit{In this paper, we propose an algorithm that secures
the key used in AES by converting key into an image and
splitting it into n shares using Visual Secret Sharing
techniques that is hardware friendly and offers backward
compatibility.}

\section{Estructura del paper}

El Paper 1 es el único que no presenta cómo esta estructurado su paper; los otros dos si contienen esta parte:\\
Paper 2:\par
\textit{The rest of the paper is organized as described. Section II
gives basic information about AES and Visual Cryptography.
Section III gives details about the proposed algorithm along.[...]} \par

Paper 3:\par
\textit{Organization of this paper is as follow: Section II
describes the related work in visual cryptography for color
image.[...]}


\end{document}
