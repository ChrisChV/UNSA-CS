\documentclass[a4paper,12pt]{article}
\usepackage[utf8]{inputenc}
\usepackage[spanish]{babel}
\usepackage{color}
\usepackage{parskip}
\usepackage{graphicx}
\usepackage{multirow}
\usepackage{listings}
\usepackage{vmargin}
\usepackage{datetime}
\newdate{date}{23}{08}{2017}
\graphicspath{ {imagenes/} }
\definecolor{mygreen}{rgb}{0,0.6,0}
\definecolor{lbcolor}{rgb}{0.9,0.9,0.9}
\usepackage{epstopdf}


\setpapersize{A4}
\setmargins{2.5cm}       % margen izquierdo
{1.5cm}                        % margen superior
{16.5cm}                      % anchura del texto
{23.42cm}                    % altura del texto
{10pt}                           % altura de los encabezados
{1cm}                           % espacio entre el texto y los encabezados
{0pt}                             % altura del pie de página
{2cm}     



\begin{document}
\title{Práctica de Empresas}
  \author{
  Christofer Fabián Chávez Carazas \\
  \small{Universidad Nacional de San Agustín} \\
  \small{Empresas I}
}
\date{\displaydate{date}}

\maketitle


\begin{itemize}
 \item \textbf{¿Cuál es la región más innovadora del mundo?}
 En mi opinión, uno de los países más innovadores es Estados Unidos. Y si hablamos de continentes, vendría ser Europa.
 
 \item \textbf{¿Áreas prioritarias en Ciencia y Tecnología en el Perú?} \\
 En estos momentos las áreas prioritarias serían las relacionadas con resolver problemas generales de la población,
 como por ejemplo, el exceso de tráfico en las grandes ciudades o los desastres naturales. También es importante el área
 administrativa del estado. Hace falta mucho software que administre las diferentes áreas del estado, para tener una mejor
 visión de lo que está pasando en el Perú en tiempo real, para luego dar lugar a que nuestros gobernantes tomen mejores decisiones.
 CONCYTEC nos presenta el plan nacional  de CTI 2006 – 2021 que tiene como objetivo asegurar la articulación y concertación entre los
 actores del SINACYT, enfocando sus esfuerzos para atender las demandas tecnológicas en áreas estratégicas prioritarias, con la finalidad de 
 elevar el valor agregado y la competitividad, mejorar la calidad de vida de la población y contribuir con el manejo responsable del medio ambiente.
 
 \item \textbf{¿Cuáles son los motivos para plantearse la creación de una empresa?} \\
 En mi opinión puede haber varios motivos por lo cual uno quiera crear una empresa. Uno de ellos, y por el que tal vez todas las
 empresas piensan, es para lucrar. Otro motivo, y tal vez el más importante, es producir un producto o servicio que resuelva 
 un problema del consumidor, o que haga que su día a día sea más fácil.
 
 \item \textbf{¿Qué elementos principales conforman una empresa?} \\
 En primer lugar está toda la organización de la empresa: gerente, ejecutivos y empleados. Luego están los conocimientos
 y tecnologías dentro de la empresa. Y ya teniendo los dos elementos anteriores recién se puede hablar de los bienes y servicios que ofrece la empresa a los
 consumidores.
 
 \item \textbf{¿Qué características posee una empresa de Base Tecnológica?} \\
 Una empresa de Base Tecnológica basan su actividad en las aplicaciones de nuevos descubrimientos científicos o tecnológicos
 para la generación de nuevos productos, procesos o servicios. Esto se logra con la investigación y desarrollo (I+D).
 
 \item \textbf{¿Tiene alguna idea de negocio a emprender?¿Cuál?} \\
 Aún no.
 
 \item \textbf{¿Tiene alguna experiencia como empresario o gerente?} \\
 No tengo experiencia como empresario o gerente.
 
\end{itemize}

\end{document}

